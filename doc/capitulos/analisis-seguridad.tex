\cleardoublepage

\chapter{Análisis de seguridad}

Una vez detallado el diseño e implementación del sistema de autenticación propuesto, es necesario conocer la caracterísiticas
de seguridad que implementa. 

\textit{Farooq} propone en \cite{farooq2019elliptic} un sistema similar al propuesto en este trabajo pero centrado en sistemas
electrónicos de tipo contador inteligentes. En él, se hace un estudio de los distintas vulnerabilidades a ciberataques que 
el sistema evita que se sean explotados.

\section{\acrfull{mitm}}

El ataque MITM conocido como ``Ataque de Hombre en el Medio'' es muy común en todo sistema que use una red pública como puede
ser Internet para establecer una comunicación entre dos o más integrantes. 
Para evitar este tipo de ataque, el sistema propuesto usa el algoritmo ECDSA para firmar el desafío y dado que solo el servidor
conoce la clave pública del cliente. En caso de una tercera persona, para que esta se pudiera pasar por el cliente
necesitaría interceptar el desafío y conocer su clave privada para firmarlo. 

\section{Integridad del desafío}

Dado que la respuesta al desafío proviene de la salida de una función hash que ha sido procesada en ambos lados, tanto cliente
como servidor, sin que llege a ser envíada por algún canal, implica que si alguna tercera persona modifica el desafío o la respuesta
a este, el sistema lo detectará y no pasará.

\section{Confidencialidad del mensaje}

Como ya se ha mencionado en \ref{sec:tls_phase}, la comunicación con el broker MQTT se hace vía TLS. Esto permite que los 
mensajes vaya cifrados por una clave. Para usar el broker MQTT y comunicarse con el servidor o viceversa, es necesario presentar
un certificado. Esto garantizar que solo usuario legítimos se comumiquen con el servidor.




