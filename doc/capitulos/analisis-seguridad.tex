\chapter{Análisis de seguridad}
\label{chap:analisis-seguridad}

Una vez detallado el diseño e implementación del sistema de autenticación propuesto, es necesario conocer la caracterísiticas
de seguridad que implementa. 

\textit{Farooq} propone en \cite{farooq2019elliptic} un sistema similar al propuesto en este trabajo pero centrado en sistemas
electrónicos de tipo contador inteligentes. En él, se hace un estudio de los distintas vulnerabilidades a ciberataques que 
el sistema evita que se sean explotados y los cuales se van a analizar contra este sistema propuesto.

\section{\acf[first-style=short]{mitm}}

El ataque MITM conocido como ``Ataque de Hombre en el Medio'' es muy común en todo sistema que use una red pública como puede
ser Internet para establecer una comunicación entre dos o más integrantes. 
Para evitar este tipo de ataque, el sistema propuesto usa el algoritmo ECDSA para firmar el desafío usando la clave privada del 
cliente. Para que una tercera persona se haga pasar por este cliente, esta primera tendría que conocer dicha clave.

\section{Integridad del desafío}

El hash del desafío se calcula en ambas parte, cliente y servidor, sin llegar a enviarse. Por ello, si alguien qu estuviera 
escuchando y modificase el desafío, el cliente fimaría un hash distinto al servidor y por tanto no se verificaría. 

\section{Confidencialidad del mensaje}

Como ya se ha mencionado en \ref{sec:tls_phase}, la comunicación con el broker MQTT se hace vía TLS. Esto permite que los 
mensajes vaya cifrados por una clave. Para usar el broker MQTT y comunicarse con el servidor o viceversa, es necesario presentar
un certificado. Esto garantizar que solo usuario legítimos se comumiquen con el servidor.




