\chapter{Análisis de seguridad}
\label{chap:analisis-seguridad}

Una vez detallado el diseño e implementación del sistema de autenticación propuesto, es necesario conocer la caracterísiticas
de seguridad que implementa. 

\section{Vulnerabilidades afrontadas}

\textit{Farooq} propone en \cite{farooq2019elliptic} un sistema similar al propuesto en este trabajo pero centrado en sistemas
electrónicos de tipo contador inteligentes. En él, se hace un estudio de los distintas vulnerabilidades a ciberataques que 
el sistema evita que se sean explotados y los cuales se van a analizar contra este sistema propuesto.

\subsection{\acf[first-style=short]{mitm}}

El ataque MITM conocido como ``Ataque de Hombre en el Medio'' es muy común en todo sistema que use una red pública como puede
ser Internet para establecer una comunicación entre dos o más integrantes. 
Para evitar este tipo de ataque, el sistema propuesto usa el algoritmo ECDSA para firmar el desafío usando la clave privada del 
cliente. Para que una tercera persona se haga pasar por este cliente, esta primera tendría que conocer dicha clave.

\subsection{Integridad del desafío}

El hash del desafío se calcula en ambas parte, cliente y servidor, sin llegar a enviarse. Por ello, si alguien qu estuviera 
escuchando y modificase el desafío, el cliente fimaría un hash distinto al servidor y por tanto no se verificaría. 

\subsection{Confidencialidad del mensaje}

Como ya se ha mencionado en \ref{sec:tls_phase}, la comunicación con el broker MQTT se hace vía TLS. Esto permite que los 
mensajes vaya cifrados por una clave. Para usar el broker MQTT y comunicarse con el servidor o viceversa, es necesario presentar
un certificado. Esto garantiza que sólo usuarios legítimos se comumiquen con el servidor.

\section{Análisis de resultados}

En la figura \ref{code:client-script} se ve la salida del script del lado del cliente. Concretamente, los logs del cliente MQTT. 
Entre medias aparecen dos mensajes: el primero indica que está suscrito el tópico \textit{pam/68263723-e928-4f71-8339-c609478f0a1a/challenge}
y el segundo que ha recibido el desafío \textit{ITOeM0joCRNR5dm.hWS5O7BaxvE8UdE7SMoPKoQck5WhhYu1di2KrBrxGsG6o76} del servidor.

Por otro lado, la figura \ref{code:ssh-request} muestra la salida de la petición SSH del cliente al servidor. El primer mensaje
aivsa de que ha encontrado el UUID del cliente y su valor concreto. A continuación muestra los logs del cliente MQTT con respecto 
a la conexión con el broker MQTT y las subscripciones a ambos tópicos, \textit{68263723-e928-4f71-8339-c609478f0a1a/pam/r} y 
\textit{68263723-e928-4f71-8339-c609478f0a1a/pam/s}, correspondientes a los valores en hexadecimal de la firma digital de 
curva elíptica. Posteriormente publica el desafío creado en \textit{pam/68263723-e928-4f71-8339-c609478f0a1a/challenge} y 
por último los mensajes de recepción. Al final aparece un mensaje de si la firma ha sido verificada o no y por tanto si el 
valor PAM devuelto es \textit{PAM\_SUCCESS} o \textit{PAM\_AUTH\_ERR}.  

En el lado del servidor, simplemente hay que añadir al archivo de configuración PAM de SSH \ref{code:pam-sshd} la siguiente 
directiva: \textit{auth required mqtt-pam.so broker.mqtt.com 8883 /etc/mosquitto/ca\_certificates/ca.crt}, donde:

\begin{itemize}
    \item \textit{auth} indica el tipo de módulo PAM a usar (autenticación)
    \item \textit{required} indica la política de ejecucion. En este caso, para que se ejecuten el resto de módulos PAM es 
    necesario que el valor devuelto sea \textit{PAM\_SUCCESS}
    \item \textit{mqtt-pam.so broker.mqtt.com 8883 /etc/mosquitto/ca\_certificates/ca.crt} indica el módulo y los parámetros. 
    Al no especificar la ruta absoluta del ejecutable, PAM busca por defecto en \textit{/lib/security/}
\end{itemize}


