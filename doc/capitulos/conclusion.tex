\chapter{Conclusión}
\label{chap:conclusion}

La propuesta de mejora en \cite{tesisIliaBlockin} ha sido el punto de partida de este proyecto basado en el desarrollo de un 
módulo PAM que integre un sistema de autenticación para el entorno de multiautenticación propuesto en \cite{multipauthpaper}. 
Este módulo esta pensado no solo para casos específicos como HPC o \textit{cloud} sino para cualquiera debido a la amplia 
granularidad y disponibilidad que ofrecen estos módulos. 
El desarrollo de este proyecto me ha permitido ganar conocimiento en el campo de la seguridad informática, concretamente en 
criptografía e infraestructuras de clave pública, así como desarrollo en C de un módulo PAM el cual nunca había experimentado.
Además, he aprendido a leer detalladamente artículos científicos y conocer a fondo las bases de un proyecto científico de alto 
nivel.

\section{Problemas afrontados}

Durante el desarrollo de este proyecto me he topado con diversos problemas. Al principio, me costó mucho entender el funcionamiento
de PAM. Tuve que investigar en profunidad para conocer bien su funcionamiento y así poder escribir mi propio módulo. 

Con respecto a la API de \textit{mosquitto}, a pesar de que la documentación \cite{mosquittoconf_2021} está bien estructurada,
tuve inconvenientes en el desarrollo de los scripts en C por la reserva de memoria que este lenguaje require.

\section{Trabajo futuro}

Como trabajo futuros se propone lo siguiente:

\begin{enumerate}
    \item Aplicar para la política \textit{relaxed} una condición dependiendo de la dirección IP de destino de la que provenga 
    la petición SSH de tal forma que si es de una institución, se habilite sin llegar a autenticarse.
    \item Llevar a cabo pruebas de estrés con varios accesos simultáneos desde mútliples direcciones IP al servidor para 
    comprobar el rendimiento del protocolo MQTT y del servidor
    \item Añadir una protección extra al script del servidor añadiéndole una política con SELinux
    \item Tal y como  se especifica en \cite{multipauthpaper}, usar una lista de control de acceso o ACL para indicar que puede
    hacer cada usuario y a qué nivel
\end{enumerate}