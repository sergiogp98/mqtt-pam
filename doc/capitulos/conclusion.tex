\chapter{Conclusión}
\label{chap:conclusion}

La propuesta de mejora en \cite{tesisIliaBlockin} ha sido el punto de partida de este proyecto basado en el desarrollo de un 
módulo PAM que integre un sistema de autenticación para el entorno de multiautenticación propuesto en \cite{multipauthpaper}. 
Este módulo esta pensado no solo para casos específicos como HPC o \textit{cloud} sino para cualquiera debido a la amplia 
granularidad y disponibilidad que ofrecen estos módulos. 
El desarrollo de este proyecto me ha permitido ganar conocimiento en el campo de la seguridad informática, concretamente en 
criptografía e infraestructuras de clave pública, así como desarrollo en C de un módulo PAM el cual nunca había experimentado.
Además, he aprendido a leer detalladamente artículos científicos y conocer a fondo las bases de un proyecto científico de alto 
nivel.

\section{Análisis resultados}

En la figura \ref{code:client-script} se ve la salida del script del lado del cliente. Concretamente, los logs del cliente MQTT. 
Entre medias aparecen dos mensajes: el primero indica que está suscrito el tópico \textit{pam/68263723-e928-4f71-8339-c609478f0a1a/challenge}
y el segundo que ha recibido el desafío \textit{ITOeM0joCRNR5dm.hWS5O7BaxvE8UdE7SMoPKoQck5WhhYu1di2KrBrxGsG6o76} del servidor.

Por otro lado, la figura \ref{code:ssh-request} muestra la salida de la petición SSH del cliente al servidor. El primer mensaje
aivsa de que ha encontrado el UUID del cliente y su valor concreto. A continuación muestra los logs del cliente MQTT con respecto 
a la conexión con el broker MQTT y las subscripciones a ambos tópicos, \textit{68263723-e928-4f71-8339-c609478f0a1a/pam/r} y 
\textit{68263723-e928-4f71-8339-c609478f0a1a/pam/s}, correspondientes a los valores en hexadecimal de la firma digital de 
curva elíptica. Posteriormente publica el desafío creado en \textit{pam/68263723-e928-4f71-8339-c609478f0a1a/challenge} y 
por último los mensajes de recepción. Al final aparece un mensaje de si la firma ha sido verificada o no y por tanto si el 
valor PAM devuelto es \textit{PAM\_SUCCESS} o \textit{PAM\_AUTH\_ERR}.  

En el lado del servidor, simplemente hay que añadir al archivo de configuración PAM de SSH \ref{code:pam-sshd} la siguiente 
directiva: \textit{auth required mqtt-pam.so broker.mqtt.com 8883 /etc/mosquitto/ca\_certificates/ca.crt}, donde:

\begin{itemize}
    \item \textit{auth} indica el tipo de módulo PAM a usar (autenticación)
    \item \textit{required} indica la política de ejecucion. En este caso, para que se ejecuten el resto de módulos PAM es 
    necesario que el valor devuelto sea \textit{PAM\_SUCCESS}
    \item \textit{mqtt-pam.so broker.mqtt.com 8883 /etc/mosquitto/ca\_certificates/ca.crt} indica el módulo y los parámetros. 
    Al no especificar la ruta absoluta del ejecutable, PAM busca por defecto en \textit{/lib/security/}
\end{itemize}

\section{Problemas afrontados}

Durante el desarrollo de este proyecto me he topado con diversos problemas. Al principio, me costó mucho entender el funcionamiento
de PAM. Tuve que investigar en profunidad para conocer bien su funcionamiento y así poder escribir mi propio módulo. 

Con respecto a la API de \textit{mosquitto}, a pesar de que la documentación \cite{mosquittoconf_2021} está bien estructurada,
tuve inconvenientes en el desarrollo de los scripts en C por la reserva de memoria que este lenguaje require.

\section{Trabajo futuro}

Como trabajo futuros se propone lo siguiente:

\begin{enumerate}
    \item Aplicar para la política \textit{relaxed} una condición dependiendo de la dirección IP de destino de la que provenga 
    la petición SSH de tal forma que si es de una institución, se habilite sin llegar a autenticarse.
    \item Llevar a cabo pruebas de estrés con varios accesos simultáneos desde mútliples direcciones IP al servidor para 
    comprobar el rendimiento del protocolo MQTT y del servidor
    \item Añadir una protección extra al script del servidor añadiéndole una política con SELinux
    \item Tal y como  se especifica en \cite{multipauthpaper}, usar una lista de control de acceso o ACL para indicar que puede
    hacer cada usuario y a qué nivel
\end{enumerate}