\addappheadtotoc

\appendix

\chapter{Archivos de configuración}

\begin{lstlisting}[style=Consola, caption={Archivo de configuración PAM}, label={code:pam_conf}]
    # Place your local configuration in /etc/mosquitto/conf.d/
    #
    # A full description of the configuration file is at
    # /usr/share/doc/mosquitto/examples/mosquitto.conf.example
    pid_file /run/mosquitto/mosquitto.pid

    log_dest file /var/log/mosquitto/mosquitto.log
    log_type all
    log_timestamp true

    include_dir /etc/mosquitto/conf.d

    listener 1883 localhost
    listener 8883

    cafile /etc/mosquitto/ca_certificates/ca.crt
    certfile /etc/mosquitto/certs/broker-mqtt.crt
    keyfile /etc/mosquitto/certs/broker-mqtt.key

    allow_anonymous true
\end{lstlisting}

\begin{lstlisting}[style=Consola, caption={Archivo de configuración de usuarios en /etc/anubis/uuid.csv}, label={code:user_conf}]
    username,uuid
    client-1,68263723-e928-4f71-8339-c609478f0a1a   
\end{lstlisting}

\begin{lstlisting}[style=Consola, caption={Archivo de configuración anubis en /etc/anubis/anubis.conf}, label={code:anubis_conf}]
    # ------------------------#
    # /etc/anubis/anubis.conf
    # ------------------------#
    #
    # NOTE
    # ----
    #
    # Configuration file for MQTT-PAM module used to authenticate via SSH
    # Use only a space between key and value
    #
    # Permitted values:
    # access_type [relax, strict]
    # ip_address 150.214.*.*
    # ------------------------#
    #
    # Format:
    # key value

    access_type relax   
\end{lstlisting}

\begin{lstlisting}[style=Consola, caption={Petición SSH cliente al servidor}, label={code:ssh-request}]
    vagrant@client-1:~$ ssh client-1@172.16.1.101
    client-1@172.16.1.101's password:
    Found UUID in user client-1: 68263723-e928-4f71-8339-c609478f0a1a
    Client server_2941 sending CONNECT
    Client server_2941 sending SUBSCRIBE (Mid: 1, Topic: 68263723-e928-4f71-8339-c609478f0a1a/pam/r, QoS: 0, Options: 0x00)
    Client server_2941 sending SUBSCRIBE (Mid: 1, Topic: 68263723-e928-4f71-8339-c609478f0a1a/pam/s, QoS: 0, Options: 0x00)
    Client server_2941 sending PUBLISH (d0, q0, r0, m2, 'pam/68263723-e928-4f71-8339-c609478f0a1a/challenge', ... (64 bytes))
    Client server_2941 received CONNACK (0)
    Client server_2941 received SUBACK
    Client server_2941 received PUBLISH (d0, q0, r0, m0, '68263723-e928-4f71-8339-c609478f0a1a/pam/r', ... (130 bytes))
    Client server_2941 received PUBLISH (d0, q0, r0, m0, '68263723-e928-4f71-8339-c609478f0a1a/pam/s', ... (130 bytes))
    Successfully verified
    Exiting...
    Client server_2941 sending DISCONNECT
    PAM OK
    Welcome to Ubuntu 20.04.2 LTS (GNU/Linux 5.4.0-74-generic x86_64)

    * Documentation:  https://help.ubuntu.com
    * Management:     https://landscape.canonical.com
    * Support:        https://ubuntu.com/advantage

    System information as of Mon Jun 21 21:19:03 UTC 2021

    System load:  0.0               Processes:               109
    Usage of /:   7.0% of 38.71GB   Users logged in:         1
    Memory usage: 41%               IPv4 address for enp0s3: 10.0.2.15
    Swap usage:   0%                IPv4 address for enp0s8: 172.16.1.101

    * Super-optimized for small spaces - read how we shrank the memory
    footprint of MicroK8s to make it the smallest full K8s around.

    https://ubuntu.com/blog/microk8s-memory-optimisation

    38 updates can be applied immediately.
    To see these additional updates run: apt list --upgradable


    Last login: Mon Jun 21 21:11:24 2021 from 172.16.1.111
    client-1@server-1:~$
\end{lstlisting}

\begin{lstlisting}[style=Consola, caption={Salida script cliente}, label={code:client-script}]
    vagrant@client-1:~/tfg/bin$ ./client broker.mqtt.com 8883 68263723-e928-4f71-8339-c609478f0a1a /etc/mosquitto/ca_certificates/ca.crt
    Client 68263723-e928-4f71-8339-c609478f0a1a sending CONNECT
    Client 68263723-e928-4f71-8339-c609478f0a1a sending SUBSCRIBE (Mid: 1, Topic: pam/68263723-e928-4f71-8339-c609478f0a1a/challenge, QoS: 0, Options: 0x00)
    Listening to pam/68263723-e928-4f71-8339-c609478f0a1a/challenge topic...
    Client 68263723-e928-4f71-8339-c609478f0a1a received CONNACK (0)
    Client 68263723-e928-4f71-8339-c609478f0a1a received SUBACK
    Client 68263723-e928-4f71-8339-c609478f0a1a received PUBLISH (d0, q0, r0, m0, 'pam/68263723-e928-4f71-8339-c609478f0a1a/challenge', ... (64 bytes))
    Received challenge: ITOeM0joCRNR5dm.hWS5O7BaxvE8UdE7SMoPKoQck5WhhYu1di2KrBrxGsG6o76
    Client 68263723-e928-4f71-8339-c609478f0a1a sending PUBLISH (d0, q0, r0, m2, '68263723-e928-4f71-8339-c609478f0a1a/pam/r', ... (130 bytes))
    Client 68263723-e928-4f71-8339-c609478f0a1a sending PUBLISH (d0, q0, r0, m3, '68263723-e928-4f71-8339-c609478f0a1a/pam/s', ... (130 bytes))
    Exiting...
    Client 68263723-e928-4f71-8339-c609478f0a1a sending DISCONNECT
\end{lstlisting}

\begin{lstlisting}[style=Consola, caption={Archivo de configuración PAM para sshd}, label={code:pam-sshd}]
    # PAM configuration for the Secure Shell service

    # MQTT PAM module
    auth required mqtt-pam.so broker.mqtt.com 8883 /etc/mosquitto/ca_certificates/ca.crt

    # Standard Un*x authentication.
    @include common-auth

    # Disallow non-root logins when /etc/nologin exists.
    account    required     pam_nologin.so

    # Uncomment and edit /etc/security/access.conf if you need to set complex
    # access limits that are hard to express in sshd_config.
    # account  required     pam_access.so

    # Standard Un*x authorization.
    @include common-account

    # SELinux needs to be the first session rule.  This ensures that any
    # lingering context has been cleared.  Without this it is possible that a
    # module could execute code in the wrong domain.
    session [success=ok ignore=ignore module_unknown=ignore default=bad]        pam_selinux.so close

    # Set the loginuid process attribute.
    session    required     pam_loginuid.so

    # Create a new session keyring.
    session    optional     pam_keyinit.so force revoke

    # Standard Un*x session setup and teardown.
    @include common-session

    # Print the message of the day upon successful login.
    # This includes a dynamically generated part from /run/motd.dynamic
    # and a static (admin-editable) part from /etc/motd.
    session    optional     pam_motd.so  motd=/run/motd.dynamic
    session    optional     pam_motd.so noupdate

    # Print the status of the user's mailbox upon successful login.
    session    optional     pam_mail.so standard noenv # [1]

    # Set up user limits from /etc/security/limits.conf.
    session    required     pam_limits.so

    # Read environment variables from /etc/environment and
    # /etc/security/pam_env.conf.
    session    required     pam_env.so # [1]
    # In Debian 4.0 (etch), locale-related environment variables were moved to
    # /etc/default/locale, so read that as well.
    session    required     pam_env.so user_readenv=1 envfile=/etc/default/locale

    # SELinux needs to intervene at login time to ensure that the process starts
    # in the proper default security context.  Only sessions which are intended
    # to run in the user's context should be run after this.
    session [success=ok ignore=ignore module_unknown=ignore default=bad]        pam_selinux.so open

    # Standard Un*x password updating.
    @include common-password
\end{lstlisting}

\begin{lstlisting}[language=Ruby, caption={Archivo de configuración Vagrantfile}, label={code:vagrantfile}]
    n_servers = 2
    n_clients = 1
    
    server_net = "172.16.1.10"
    client_net = "172.16.1.11"
    broker_mqtt_net = "172.16.1.100"
    
    def add_ssh_key(config)
        ssh_pub_key = File.readlines("#{Dir.home}/.ssh/id_rsa.pub").first.strip
        config.vm.provision "shell" do |s|
          s.inline = <<-SHELL
            echo #{ssh_pub_key} >> /home/vagrant/.ssh/authorized_keys
            mkdir -p /root/.ssh/
            chmod 700 /root/.ssh/
            echo #{ssh_pub_key} >> /root/.ssh/authorized_keys
          SHELL
        end
    end
    
    Vagrant.configure("2") do |config|
        config.vm.synced_folder ".", "/home/vagrant/tfg"
    
        (1..n_servers).each do |i|
            config.vm.define "server-#{i}" do |node|
                node.vm.network :private_network, ip: "#{server_net}#{i}"
                node.vm.box = "ubuntu/focal64"
                node.vm.provider "virtualbox" do     |pmv|
                    pmv.memory = 512
                    pmv.cpus   = 1
                end
                node.vm.hostname = "server-#{i}"
                add_ssh_key(config)
            end
        end
    
        (1..n_clients).each do |i|
            config.vm.define "client-#{i}" do |node|
                node.vm.network :private_network, ip: "#{client_net}#{i}"
                node.vm.box = "ubuntu/focal64"
                node.vm.provider "virtualbox" do |pmv|
                    pmv.memory = 512
                    pmv.cpus   = 1
                end
                node.vm.hostname = "client-#{i}"
                add_ssh_key(config)
            end
            
        end
    
        config.vm.define "broker-mqtt" do |node|
            node.vm.network :private_network, ip: "#{broker_mqtt_net}"
            node.vm.box = "ubuntu/focal64"
            node.vm.provider "virtualbox" do |pmv|
                pmv.memory = 512
                pmv.cpus   = 1
            end
            node.vm.hostname = "broker-mqtt"
            add_ssh_key(config)
        end
    end        
\end{lstlisting}
