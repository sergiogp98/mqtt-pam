\chapter{Introducción}

El concepto de ``Seguridad de la Información'' se acuñó por primera vez en un artículo del Instituo Nacional de Estándares y 
Teconología (\acf[first-style=short]{nist}) en 1997 \cite{neumann1977post} aunque la seguridad en el campo de la tecnología siempre ha estado en 
boca de todos y ha sido un apsecto a tener en cuenta. 

La ciberseguridad se empezó a tener en cuenta con el ``\textit{boom}'' de la era digital. No obstante, habría que remontarse unos 
siglos atras para saber cuando se empezó a aplicar métodos seguros a la información. Concretamente cuando Julio César vivía, se 
empezó a usar algoritmos de cifrado simples que se aplicaban sobre mensajes que tenían que ser enviados y así si alguien lo 
interceptada, no podía saber su contenido. A esta práctica se conoce como criptografía. La criptografía ha estado latente en muchos 
hitos de nuestra historia, como por ejemplo con la aparición Alang Turing y su ingenio para descifrar las comunicaciones entre 
integrantes del ejército alemán en código morse.

La criptografía, según la \acf[first-style=short]{rae} es el ``arte de escribir con clave secreta o de un modo enigmático''. Esta herramienta
ha sido la base de la seguriadad de la información ya que sustenta las tres propiedades de todo sistema seguro (CIA): confidencialidad 
(\textit{confidenciality}), integridad (\textit{integrity}) y disponibilidad (\textit{availability}).

Tal y como se puede apreciar en \cite{digital_attack}, el número de ciberataques que suceden a diaría es muy elevado. Es por ello
que se cualquier empresa que de soporte a clientes almacenando información sensible use herramientas de seguridad que estén al 
día de los distintas vulnerabilidades que se descubren \cite{cve_mitre}.

\section{Motivación}
\label{sec:motivacion}

La idea de este tarbajo nace de a raiz de la tesis doctoral \cite{tesisIliaBlockin} publicada en 2020 por la Universidad de Granada
titulada ``\textit{Mecanismos de seguridad para Big Data basados en circuitos criptográficos}'' y elaborada por \textit{Ilia 
Blockin}. Esta tesis sugiere soluciones basada en sistemas electrónicos que ofrecen una solución eficiente y flexible para 
aumentar la seguridad en el acceso a servicios y sistemas que pueden procesar gran cantidad de información. 

El planteamiento se detalla en unos de las mejoras propuestas de dicha tesis: ``\textit{continuar mejorando los sistemas propuestos 
agregando compatibilidad con otros métodos de autenticación como el Módulo de autenticación conectable de Linux (PAM) u otros 
protocolos de autenticación}'' 

Personalmente, me he decantado por la elección de este tema ya que tengo especial interés en el campo de la ciberseguridad y por 
el reto de conocer en profundidad las bases de la criptografía.

\section{Objetivos}

Para elaborar la lista de objetivos, es necesario conocer los requisitos del sistema donde se pretende implantar. Este entorno
viene descrito de forma detallada en \cite{multipauthpaper}. 

\subsection{Diseño del sistema propuesto}

Se propone un sistema con un conjunto de elementos físicos y virtualizados:

\begin{itemize}
    \item \textit{Usurio}: quien accede al sistema
    \item \textit{Servicios}: ofrecido por programas a través de conexión TCP 
    \item \textit{Servidores}: ordenadores donde los servicios son instalados
    \item \textit{Clientes}: ordenadores desde donde los servicios son accedidos
    \item \textit{eToken}: dispositivos que autentica la petición del usuario
\end{itemize}

Se define una lista de control (\acf[first-style=short]{acl}) con reglas para garantizar una autorización deseada. Cada elemento viene 
identificado por un Identificador Único Universal (\acf[first-style=short]{uuid}) de 128 bits y un par de claves pública y privada que permite 
al sistema a autenticar las peticiones.
El sistema es híbrido, es decir que usan la clave pública combinado con un modelo centralizado para autorizar el acceso. Hay 
dos elementos más:

\begin{itemize}
    \item \textit{Servidor de configuración}: permite modificar la configuración de cada elemento remotamente
    \item \textit{Servidor de autenticación}: autoriza la petición de acceso a servicios basado en las ACL
    \item \textit{Gateway}: redirige la conexión de red entre clinte y servidores
    \item \textit{Broker MQTT}: servidor de intercambio de mensajes 
\end{itemize}

\subsection{Listado de objetivos}

El presente trabajo conlleva el cumplimiento de los siguientes objetivos:

\begin{enumerate}
    \item Crear un módulo \acf[first-style=short]{pam} para el sistema de autenticación propuesto en \cite{multipauthpaper}
    \item Implementar el módulo PAM para el servicio \acf[first-style=short]{ssh}
    \item Usar el protocolo \acf[first-style=short]{mqtt}
    \item Seguir esquema de autenticación \textit{challenge-response} \cite{newman2010salted}
    \item Cifrar el \textit{challenge} mediante un algoritmo de cifrado unidireccional robusto como por ejemplo SHA512 
    (\acf[first-style=short]{sha})
    \item Usar Criptografía de Curva Elíptica o \acf[first-style=short]{ecc} tanto para la firma del \textit{challenge} como para la verificación 
    del mismo usando el Algoritmo de Firma Digital de Curva Elíptica (\acf[first-style=short]{ecdsa})
    \item Encriptar las comunicaciones entre los elementos del sistema propuesto usando \acf[first-style=short]{tls} versión 1.2
\end{enumerate}

\section{Estructura del trabajo}

El presente trabajo se divide en las siguientes capítulos: en el Capítulo \ref{chap:estado-arte} se detalla la ``situación actual de la 
tecnología''. Se habla de otros proyectos que existen actualmente y que realizan funcionalidades iguales o parecidas a las 
que se propone en este proyecto. Se valora los puntos fuertes y ámbitos en los que se puede aplicar. En el Capítulo \ref{chap:analisis-problema},
se habla de la seguridad en el apartado de la autenticación en sistemas, la seguridad aplicada a entornos IoT, el protocolo
MQTT usado, la criptografía de curva elíptica y el algoritmo ECDSA y por último PAM y su arquitectura. Al final se hace alusión
de las herramientas usadas para el desarrollo de este trabajo. En el Capítulo \ref{chap:diseño} se hace un análisis exhaustivo del 
funcionamiento del programa indicando sus distintas fases. En el Capítulo \ref{chap:analisis-seguridad} lista algunas caracterísiticas
de seguridad y cómo las lleva a cabo. En el Capítulo \ref{chap:presupuesto} se hace un resumen de lo que ha costado económicamente el 
proyecto y su desarrollo temporal con un diagrama de Gantt. Por último, en el Capítulo \ref{chap:conclusion} se hace un resumen de objetivo
de este proyecto, los resultado obtenidos, problemas afrontados y futuras mejoras.