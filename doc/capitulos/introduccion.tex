\cleardoublepage

\chapter{Introducción}

El concepto de ``Seguridad de la Información'' se acuño por primera vez mencionado por el NIST en 1997 \cite{neumann1977post}
aunque la seguridad en el campo de la tecnología siempre ha estado en boca de todos y ha sido un apsecto a tener en cuenta. 

La seguridad en el campo de la tecnología para un campo que se empezó a tener en cuenta con el ``\textit{boom}'' de la 
tecnología. No obstante, habría que remontarse unos siglos atras, en concreto cuando Julio César vivía para darse cuenta que en 
aquella época ya se aplicaban ciertas metodologías para envíar mensajes sin que otros se enterasen. Exactamente, hace más de 
2000 años pusieron en prática la técnica de comunicación entre entidades de forma segura a través de un canal inseguro, la 
criptografía. La criptografía ha estado latente en muchos hitos de nuestra historia, como por ejemplo con la aparición Alang 
Turing y su ingenio para descifrar las comunicaciones entre integrantes del ejército alemán en código morse.

La criptografía, según la \acrfull{rae} es el ``arte de escribir con clave secreta o de un modo enigmático''. Esta herramienta
ha sido la base de la seguriadad de la información ya que sustenta las bases de todo sistema serguro (CIA): confidencialidad 
(\textit{confidenciality}), integridad (\textit{integrity}) y disponibilidad (\textit{availability}).

La seguridad es un campo de elevada atracción hacia los investigadores y es por ello que la demanda de sistemas altamente seguros
es un requisito indispensable. A esto, hay que añadirle la posibilidad de tener a un clic acceso a toda información que queramos.
Esto supone un mayor esfuerzo a la hora de clasificar que información es más sensible que otra y por tanto que grado de seguridad
hay que aplicar. 

Dada que toda la información no tiene la misma validez, cualquier sistema que envíe tráfico hacia internet con información sensible
debe cumplir unos requisitos de seguridad necesarios para que no sea intercedida ni modificada. La empresas invierten cada año
más en seguridad dado el incremento de ataques informáticos que suceden a diario, tal y como se puede ver 
en esta página \cite{digital_attack} y la aparición constante de nuevas vulnerabilidades, las cuales se pueden consultar 
\cite{cve_mitre}.

\section{Motivación}
\label{sec:motivacion}

La idea de este tarbajo nace de raiz de la tesis doctoral \cite{tesisIliaBlockin} publicada en 2020 por la Universidad de Granada
titulada ``\textit{Mecanismos de seguridad para Big Data basados en circuitos criptográficos}'' y elaborada por \textit{Ilia 
Blockin}. Esta tesis sugiere soluciones basada en sistemas electrónicos que permitan mejorar la seguridad en el acceso a sistemas 
donde deben procesarse un elevado volumen de datos. Los sistemas propuestos ofrecen una solución eficiente y flexible para 
aumentar la seguridad en el acceso a servicios y sistemas que pueden procesar gran cantidad de información. 

El planteamiento se detalla en unos de las mejoras propuestas de dicha tesis: ``\textit{continuar mejorando los sistemas propuestos 
agregando compatibilidad con otros métodos de autenticación como el Módulo de autenticación conectable de Linux (PAM) u otros 
protocolos de autenticación}'' 

Personalmente, me he decantado por la elección de este tema ya que tengo especial interés en el campo de la ciberseguridad y 
lo importante que es a día de hoy en el desarrollo de cualquier sistema y producto software.

Considero que este trabajo es relevante dado el incremento de popularidad que se está dando en el campo del \textit{Big Data} y 
la importancia de que los \acrfull{cpd} que se usan estén bien protegidos y solo el personal autorizado tenga acceso a ellos.

\section{Objetivos}

Para elaborar la list de objetivos, es necesario conocer los requisitos del sistema donde se pretende implantar. Este entorno
viene descrito de forma detallada en \cite{multipauthpaper}. 

El presente trabajo conlleva el cumplimiento de los siguientes objetivos:

\begin{enumerate}
    \item Crear un \acrfull{pam} para el sistema de autenticación propuesto en \cite{multipauthpaper}
    \item Implementar el módulo PAM para el servicio \acrfull{ssh}
    \item Usar el protocolo \acrfull{mqtt}
    \item Seguir esquema de autenticación \textit{challenge-response} \cite{newman2010salted}
    \item Cifrar el \textit{challenge} mediante el algoritmo de cifrado unidireccional robusto como por ejemplo SHA512 
    (\acrlong{sha})
    \item Usar Criptografía de Curva Elíptica o \acrfull{ecc} tanto para la firma del \textit{challenge} como para la verificación del mismo 
    usando el algoritmo \acrfull{ecdsa}
    \item Encriptar las comunicaciones entre los elementos del sistema propuesto usando \acrfull{tls} versión 1.2
\end{enumerate}

\section{Estructura del trabajo}

El presente trabajo se divide en las siguientes secciones y subsecciones: ....

\section{Convenciones}

El texto de este trabajo mantiene una serie de reglas de cumplimiento de texto:
 
\begin{itemize}
    \item Los acrónimos se describen la primera vez que aparezcan en su idioma original seguido del acrónimo abreviado entre 
    paréntesis. Posteriormente, solo se usará el acrónimo abreviado.
    \item Las palabras en otro idioma se detallan en cursiva.
\end{itemize}