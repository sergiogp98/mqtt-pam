%------------------ KEY WORDS - ABSTRACT (ESP) ------------------------%

\chapter*{}

\thispagestyle{empty}

\cleardoublepage

\begin{center}
    {\large\bfseries \myTitle}\\
\end{center}

\begin{center}
    \myName\\
\end{center}

\noindent{\textbf{Palabras clave}: \myKeywords}\\

\vspace{0.7cm}

\noindent{\textbf{Resumen}}\\

Cada vez está más en boca el término ``seguridad''. No es de extrañar que desde que comenzó la era digital, los retos tecnológicos son cada vez más complejos. No obstante, dado el grado 
de conocimiento y herramienta de las que disponemos, estos son bastante factibles y llamativos para cualquiera. Es por eso que el uso de elementos tecnológicos en nuestro día a día se
concibe como algo innato en nuestras vidas. 

La tecnología nos trae numeros beneficios que vale simplemente con echar la vista atrás y ver como la sociedad ha llegado a conseguir retos tan importantes gracias a la ayuda de esta. 
Sin embargo, la tecnología tiene su lado adverso y es la seguridad. Por muy llamativa y útil que pueda ser una tecnología, si no se lleva a cabo unos cumplimientos de seguridad bien definidos
y estrictos durante su desarrollo, el usuario final puede no llegar a usarlo y por tanto quedar en vano el trabajo realizado.

Gracias a la era del ``Big Data'' en la que vivmos, tenemos acceso a una ingesta cantidad de information al instante la cual se estudia y se llevan a cabo análisis complejos sobre la misma. 
Esta característica asociada a la seguridad hace que el conocimiento acerca de las posibles vulnerabilidades sea un campo que nunca parace acabar. Es por eso que todo sistema no es 100\%
seguro.

El objetivo de este trabajo es el estudio de la seguridad en entornos definidos (HPC y Cloud) donde el volumen de datos es elevado y por consiguiente el almacenamiento y manejo del mismo
se tiene que hacer desarrollar de forma segura y garantizando las bases de todo sistema seguro: confidencialidad, integridad y disponibilidad. 
Se propone una solución basada en un sistema electrónico que garantiza una autenticación a estos entornos de forma más segura. 

Este estudio se centra principalmente en la incorporación de un modulo de seguridad a este sistema de forma que pueda ser usado por cualquier otro dispositivo en entornos variados 
garantizando una solución portable y eficiente.

%------------------ KEY WORDS - ABSTRACT (EN) ------------------------%

\thispagestyle{empty}

\begin{center}
    {\large\bfseries \myTitleEN}\\
\end{center}

\begin{center}
    \myName\\
\end{center}

\noindent{\textbf{Keywords}: \myKeywordsEN}\\

\vspace{0.7cm}

\noindent{\textbf{Abstract}}\\


The term ``security'' is becoming more and more popular. It is not surprising that since the digital era began, technological challenges have become increasingly complex. However, given the degree of knowledge and tools at our disposal, they are quite feasible and appealing to anyone. That is why the use of technological elements in our day-to-day life is conceived as something innate in our lives. 

Technology brings us so many benefits that it is worth just looking back and seeing how society has achieved such important challenges thanks to its help. 
However, technology has its downside and that is security. No matter how flashy and useful a technology may be, if strict and well-defined security standards are not followed during its development, the end user may not be able to use it and therefore the work done may be in vain.

Thanks to the ``Big Data'' era in which we live, we have access to an instantaneous intake of information which is studied and complex analyses are carried out on it. 
This characteristic associated with security means that knowledge about possible vulnerabilities is a field that never seems to end. That is why every system is not 100\% secure.

The objective of this work is the study of security in defined environments (HPC and Cloud) where the volume of data is high and therefore the storage and management of the same must be developed in a secure way and guaranteeing the bases of any secure system: confidentiality, integrity and availability.  We propose a solution based on an electronic system that guarantees a more secure authentication to these environments. 

This study is mainly focused on the incorporation of a security module to this system so that it can be used by any other device in different environments, guaranteeing a portable and efficient solution. 

%------------------ AUTORIZACION ------------------------% 

\chapter*{}

\thispagestyle{empty}

\noindent\rule[-1ex]{\textwidth}{2pt}\\[4.5ex]

Yo, \textbf{\myName}, alumno de la titulación \myDegree de la \textbf{\myFaculty de la \myUni}, con DNI 77148519X, autorizo la
ubicación de la siguiente copia de mi Trabajo Fin de Grado en la biblioteca del centro para que pueda ser
consultada por las personas que lo deseen.

\vspace{6cm}

\noindent Fdo: \myName

\vspace{2cm}

\begin{flushright}
Granada a \myTime
\end{flushright}

\chapter*{}

\thispagestyle{empty}

\noindent\rule[-1ex]{\textwidth}{2pt}\\[4.5ex]

D. \textbf{\myProf}, Profesor del \myDepartment de la \myUni.

\vspace{0.5cm}

\textbf{Informan:}

\vspace{0.5cm}

Que el presente trabajo, titulado \textit{\textbf{\myTitle}}, ha sido realizado bajo su supervisión por \textbf{\myName}, y autorizamos la defensa de dicho trabajo ante el tribunal
que corresponda.

\vspace{0.5cm}

Y para que conste, expiden y firman el presente informe en Granada a \myTime.

\vspace{1cm}

\textbf{El director:}

\vspace{5cm}

\noindent \textbf{\myProf}

%------------------ AGRADECIMIENTOS ------------------------% 

\chapter*{Agradecimientos}

\thispagestyle{empty}

\vspace{1cm}

A mi director de Trabajo Fin de Grado, \myProf  por su coordinación y conocimiento acerca de la materia que me ha ayudado a desarrollar este trabajo a pesar de los inconvenientes.

Tambien me gustaría agradecer este trabajo al apoyo de mi madre y hermanos que han confiado en mi en todo momento y me han dado soporte en los momentos mas dificiles de
mi recorrido académico. Pero sobre todo a mi padre, que ha sido mi apoyo incondicional en estos años de carrera y del que seguro que estaría orgulloso de ver lo lejos que he llegado.

¡Gracias!
