%------------------ KEY WORDS - ABSTRACT (ESP) ------------------------%

\thispagestyle{empty}

\cleardoublepage

\begin{center}
    {\large\bfseries \myTitle}\\
\end{center}

\begin{center}
    \myName\\
\end{center}

\noindent{\textbf{Palabras clave}: \myKeywords}\\

\vspace{0.7cm}

\noindent{\textbf{Resumen}}\\

Cada vez está más en boca el término ``seguridad''. No es de extrañar que desde que comenzó la era digital, los retos tecnológicos 
son cada vez más complejos. No obstante, dado el grado de conocimiento y herramienta de las que disponemos, estos son bastante 
factibles y llamativos para cualquiera. Es por eso que el uso de elementos tecnológicos en nuestro día a día se
concibe como algo innato en nuestras vidas. 

La tecnología nos trae numerosos beneficios. Nada más que con echar la vista atrás y ver como la sociedad ha llegado a conseguir 
retos tan importantes gracias a la ayuda de esta es suficiente. Sin embargo, la tecnología tiene su lado adverso, la seguridad. 
Por muy llamativa y útil que pueda ser una tecnología, si no se lleva a cabo unos cumplimientos de seguridad bien definidos
y estrictos durante su desarrollo, el usuario final puede no llegar a usarlo y por tanto quedar en vano el trabajo realizado o 
peor aun, que sea usado muchos usuarios y tengas graves vulnerabilidades que afecten a la integridad de sus datos.

Con la llegada del \textit{Big Data}, cada vez más tenemos acceso a cualquier información en cualquier momento. Las compañías 
hacen un arduo esfuerzo en proteger nuestros datos y que estos no caigan en las manos equivocadas. Pero como bien es de saber, 
ningún sistema, por muy complejo y sofisticado que pueda llegar a ser, es 100\% seguro. Cada día salen nuevas vulnerabilidades
y complejos ataques para corromper estos sistemas y es por ello que la inversión de las compañías en ciberseguridad incrementa
cada año. Para que esta inversión sea eficiente, es necesario apostar por la investigación en este campo y así garantizar un 
futuro más ``seguro''.

El objetivo de este trabajo es el estudio de la seguridad en entornos definidos (\acf[first-style=short]{hpc} y Cloud) donde el volumen de datos 
que se almacena es elevado y por consiguiente puede ser información más sensible, garantizando la confidencialidad, integridad y 
disponibilidad del sistema.  

Se propone una solución \cite{multipauthpaper} de un modelo de aunteticaión multifactor, \acf[first-style=short]{mfa}, basada en un sistema 
electrónico (MCU basado en un ESP-32) garantizando un alto rendimiento y eficiencia debido al consumo de estos componentes electrónicos,
una arquitectura distribuida ofreciendo una mayor escalabilidad y con un alta granularidad en cuanto a seguridad se refiere. 

Este trabajo se centra principalmente en la incorporación de un modulo de seguridad a este sistema de autenticación propuesto de 
forma que pueda ser usado por cualquier otro dispositivo en entornos variados garantizando una solución portable y eficiente.

%------------------ KEY WORDS - ABSTRACT (EN) ------------------------%

\thispagestyle{empty}

\cleardoublepage

\begin{center}
    {\large\bfseries \myTitleEN}\\
\end{center}

\begin{center}
    \myName\\
\end{center}

\noindent{\textbf{Keywords}: \myKeywordsEN}\\

\vspace{0.7cm}

\noindent{\textbf{Abstract}}\\


The term ``security'' is becoming more and more popular. It is not surprising that since the digital era began, technological 
challenges have become increasingly complex. However, given the degree of knowledge and tools at our disposal, they are quite 
feasible and appealing to anyone. That is why the use of technological elements in our daily lives is conceived as something 
innate in our lives. 

Technology brings us numerous benefits. Just looking back and seeing how society has achieved such important challenges thanks to 
its help is enough. However, technology has its downside: security. No matter how attractive and useful a technology may be, if it 
does not comply with well-defined and strict security standards during its development, the end user may not use it and therefore 
the work done may be in vain, or worse still, it may be used by many users and have serious vulnerabilities that affect the 
integrity of their data.

With the advent of \textit{Big Data}, we increasingly have access to any information at any time. Companies make an arduous effort 
to protect our data from falling into the wrong hands. But as we all know, no system, no matter how complex and sophisticated it 
may be, is 100\% secure. Every day new vulnerabilities and complex attacks to corrupt these systems emerge and that is why 
companies' investment in cybersecurity increases every year. In order for this investment to be efficient, it is necessary to 
invest in research in this field and thus guarantee a more ``secure'' future.

The objective of this work is the study of security in defined environments (\acf[first-style=short]{hpc} and Cloud) where the volume of data 
stored is high and therefore may be more sensitive information, ensuring the confidentiality, integrity and availability of the 
system.  

A solution \cite{multipauthpaper} of a multifactor authentication \acf[first-style=short]{mfa} model is provided, based on an electronic system 
(MCU based on ESP-32) guaranteeing high performance and efficiency due to the consumption of these electronic components, a distributed 
architecture offering greater scalability and with a high granularity in terms of security. 

This work is mainly focused on the incorporation of a security module to this proposed authentication system so that it can be 
used by any other device in varied environments guaranteeing a portable and efficient solution.

%------------------ AUTORIZACION ------------------------% 

\thispagestyle{empty}

\cleardoublepage

\noindent\rule[-1ex]{\textwidth}{2pt}\\[4.5ex]

Yo, \textbf{\myName}, alumno de la titulación \myDegree de la \textbf{\myFaculty de la \myUni}, con DNI 77148519X, autorizo la
ubicación de la siguiente copia de mi Trabajo Fin de Grado en la biblioteca del centro para que pueda ser
consultada por las personas que lo deseen.

\vspace{6cm}

\noindent Fdo: \myName

\vspace{2cm}

\begin{flushright}
Granada a \myTime
\end{flushright}

\chapter*{}

\thispagestyle{empty}

\noindent\rule[-1ex]{\textwidth}{2pt}\\[4.5ex]

D. \textbf{\myProf}, Profesor del \myDepartment de la \myUni.

\vspace{0.5cm}

\textbf{Informan:}

\vspace{0.5cm}

Que el presente trabajo, titulado \textit{\textbf{\myTitle}}, ha sido realizado bajo su supervisión por \textbf{\myName}, y autorizamos la defensa de dicho trabajo ante el tribunal
que corresponda.

\vspace{0.5cm}

Y para que conste, expiden y firman el presente informe en Granada a \myTime.

\vspace{1cm}

\textbf{El director:}

\vspace{5cm}

\noindent \textbf{\myProf}

%------------------ AGRADECIMIENTOS ------------------------% 

\thispagestyle{empty}

\cleardoublepage

\chapter*{Agradecimientos}

\vspace{1cm}

A mi director de Trabajo Fin de Grado, \myProf  por su coordinación y conocimiento acerca de la materia que me ha ayudado a 
desarrollar este trabajo a pesar de los inconvenientes.

Tambien me gustaría agradecer este trabajo al apoyo de mi madre y hermanos que han confiado en mi en todo momento y me han dado 
soporte en los momentos mas dificiles de mi recorrido académico. Pero sobre todo a mi padre, que ha sido mi apoyo incondicional en 
estos años de carrera y del que seguro que estaría orgulloso de ver lo lejos que he llegado.

¡Gracias!
